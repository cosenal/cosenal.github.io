\documentclass[]{article}
\usepackage[T1]{fontenc}
\usepackage{lmodern}
\usepackage{fullpage}
\usepackage{amssymb,amsmath}
\usepackage{ifxetex,ifluatex}
\usepackage{fixltx2e} % provides \textsubscript
% use microtype if available
\IfFileExists{microtype.sty}{\usepackage{microtype}}{}
\ifnum 0\ifxetex 1\fi\ifluatex 1\fi=0 % if pdftex
  \usepackage[utf8]{inputenc}
\else % if luatex or xelatex
  \usepackage{fontspec}
  \ifxetex
    \usepackage{xltxtra,xunicode}
  \fi
  \defaultfontfeatures{Mapping=tex-text,Scale=MatchLowercase}
  \newcommand{\euro}{€}
\fi
\ifxetex
 \usepackage[setpagesize=false, % page size defined by xetex
             unicode=false, % unicode breaks when used with xetex
             xetex]{hyperref}
\else
 \usepackage[unicode=true]{hyperref}
\fi
\hypersetup{breaklinks=true,
%            bookmarks=true,
            pdfauthor={Alessandro Cosentino},
            pdftitle={CV},
            colorlinks=false,
            urlcolor=blue,
            linkcolor=magenta,
            pdfborder={0 0 0}}
\setlength{\parindent}{0pt}
\setlength{\parskip}{6pt plus 2pt minus 1pt}
\setlength{\emergencystretch}{3em}  % prevent overfull lines
\setcounter{secnumdepth}{0}

\title{Curriculum Vitae}
\author{Alessandro Cosentino}
\date{\today}

\usepackage{fancyhdr}
%\setlength{\headheight}{15.2pt}
\renewcommand{\headrulewidth}{0pt}% Remove header rule
\fancyhead{}
\pagestyle{fancy}
\cfoot{Alessandro Cosentino's CV -- \thepage}

\usepackage{bibentry}

\usepackage{array, xcolor}
\definecolor{lightgray}{gray}{0.8}
\newcolumntype{L}{>{\raggedleft}p{0.05\textwidth}}
\newcolumntype{R}{p{0.8\textwidth}}
\newcommand\VRule{\color{lightgray}\vrule width 0.5pt}

\begin{document}
\maketitle
\thispagestyle{empty}

\section{Education}\label{education}

\subsection{Ph.D.~in Computer Science}\label{ph.d.in-computer-science}

\emph{University of Waterloo} --- \emph{January '10 -- present}

Expected graduation date: April '15

Fellow of the \textbf{Institute for Quantum Computing}

Recipient of a \textbf{David R. Cheriton Graduate Scholarship},\\awarded
annually to forty to seventy-five full-time University of Waterloo
Computer Science graduate students on the basis of scholastic excellence
and evidence of research potential.

\subsection{M.Math in Computer
Science}\label{m.math-in-computer-science}

\emph{University of Pisa} --- \emph{February '09}

Final score: 110/110 \emph{cum laude}

\subsection{B.Math in Computer
Science}\label{b.math-in-computer-science}

\emph{University of Pisa} --- \emph{July '06}

Final score: 110/110 \emph{cum laude}

\section{Research Interests}

My research is in theoretical computer science, more specifically in quantum 
information theory.

The topic of my PhD thesis is on \emph{quantum state distinguishability} 
and the application of 
\emph{convex programming} to answer several fundamental open questions in 
\emph{quantum information theory}.

I am also interested in structural complexity theory, both classical and 
quantum.

% \bibliographystyle{plain}
% \nobibliography{publications}
\section{Publications}\label{publications}
% \begin{itemize}
%   \item \bibentry{Cosentino2014a}
%   \item \bibentry{Bandyopadhyay2014}
%   \item \bibentry{Cosentino2014}
%   \item \bibentry{Cosentino2013a}
%   \item \bibentry{Cosentino2013}
%   \item \bibentry{Cosentino2009}
% \end{itemize}

\begin{itemize}
\item Alessandro Cosentino, Robin Kothari, and John Watrous. Query-witness 
tradeoffs in quantum computation. manuscript, November 2014
\item Somshubhro Bandyopadhyay, Alessandro Cosentino, Nathaniel Johnston, 
Vincent Russo, John Watrous, and Nengkun Yu. Limitations on separable 
measurements by convex optimization. preprint,
arXiv:1408.6981v1 [quant-ph], August 2014
\item Alessandro Cosentino and Vincent Russo. Small sets of locally 
indistinguishable orthogonal maximally
entangled states. \emph{Quantum Information \& Computation}, 
14(13&14):1098–1106, 2014
\item Alessandro Cosentino, Robin Kothari, and Adam Paetznick. 
Dequantizing Read-once Quantum Formulas.
\emph{In Simone Severini and Fernando Brandao, editors, 8th Conference on the 
Theory of Quantum Computation, Communication and Cryptography (TQC 2013), 
volume 22 of Leibniz International Proceedings
in Informatics (LIPIcs)}, pages 80--92, Dagstuhl, Germany, 2013. Schloss 
Dagstuhl–Leibniz-Zentrum fuer Informatik
\item Alessandro Cosentino. Positive-partial-transpose-indistinguishable states 
via semidefinite programming.
\emph{Phys. Rev. A}, 87:012321, Jan 2013
\item Alessandro Cosentino and Simone Severini. Weight of quadratic forms and 
graph states. \emph{Phys. Rev. A}, 80:052309, Nov 2009
\end{itemize}

\section{Exchange programs}\label{exchange-programs}

\subsection{Research intern}\label{research-intern}

\emph{LIAFA, Université Paris Diderot} --- \emph{February '13 -- April
'13}

\subsection{\href{http://en.wikipedia.org/wiki/Erasmus_Programme}{Erasmus}
scholar}\label{erasmus-scholar}

\emph{Aarhus University} --- \emph{September '07 -- March '08}

\section{Teaching Experience}\label{teaching-experience}

\subsection{Teaching assistant}\label{teaching-assistant}

\emph{University of Waterloo} --- \emph{2010 -- 2013}

\begin{itemize}
\item Theory of Quantum Information (graduate course)
\item Data Structures and Data Management
\item Algorithms
\item Introduction to Computer Science
\end{itemize}

\subsection{UNIX Consultant}
\emph{Math Faculty Computing Facility, University of Waterloo} ---
\emph{Winter '12} and \emph{Fall '12}


\section{Other Experience}\label{work-experience}

\begin{description}
\item[\href{https://gnome.org/opw/}{Outreach Program for Women}
org coordinator and mentor] -- \emph{Summer '14}
\item[\href{https://www.google-melange.com/gci/homepage/google/gci2012}{Google
Code-in} org administrator] -- \emph{Winter '12}
\item[Google Summer of Code student developer] -- \emph{Summer '12}
\item[Season of KDE student developer]  -- \emph{Summer '11}
\end{description}

\section{Professional Service}\label{professional-service}

\subsection{Technical Reviewer} 
\begin{itemize}
  \item \emph{APS Physical Review A}
  \item \emph{IEEE Transactions on Information Theory}
  \item \emph{XVII Conference on Quantum Information Processing (2014)}.
\end{itemize}

\section{Technical Skills}\label{technical-skills}

\begin{itemize}
\itemsep1pt\parskip0pt\parsep0pt
\item
  Research tools: LaTeX, MATLAB/Octave and framework CVX for convex
  programming;
\item
  Web programming Languages: PHP, HTML5, Javascript, CSS;
\item
  Other technologies: RSS and Atom standards, Git.
\end{itemize}

\end{document}
