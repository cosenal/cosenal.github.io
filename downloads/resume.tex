\documentclass[]{article}
\usepackage[T1]{fontenc}
\usepackage{lmodern}
\usepackage{fullpage}
\usepackage{amssymb,amsmath}
\usepackage{ifxetex,ifluatex}
\usepackage{fixltx2e} % provides \textsubscript
% use microtype if available
\IfFileExists{microtype.sty}{\usepackage{microtype}}{}
\ifnum 0\ifxetex 1\fi\ifluatex 1\fi=0 % if pdftex
  \usepackage[utf8]{inputenc}
\else % if luatex or xelatex
  \usepackage{fontspec}
  \ifxetex
    \usepackage{xltxtra,xunicode}
  \fi
  \defaultfontfeatures{Mapping=tex-text,Scale=MatchLowercase}
  \newcommand{\euro}{€}
\fi
\ifxetex
  \usepackage[setpagesize=false, % page size defined by xetex
              unicode=false, % unicode breaks when used with xetex
              xetex]{hyperref}
\else
  \usepackage[unicode=true]{hyperref}
\fi
\hypersetup{breaklinks=true,
            bookmarks=true,
            pdfauthor={Alessandro Cosentino},
            pdftitle={Résumé},
            colorlinks=false,
            urlcolor=blue,
            linkcolor=magenta,
            pdfborder={0 0 0}}
\setlength{\parindent}{0pt}
\setlength{\parskip}{6pt plus 2pt minus 1pt}
\setlength{\emergencystretch}{3em}  % prevent overfull lines
\setcounter{secnumdepth}{0}

\title{Résumé}
\author{Alessandro Cosentino}
\date{2015-02-18}

% \usepackage{fancyhdr}
% \setlength{\headheight}{15.2pt}
% \pagestyle{fancy}

\begin{document}
\maketitle

\section{Education}\label{education}

\subsection{Ph.D.~in Computer Science}\label{ph.d.in-computer-science}

\emph{University of Waterloo} --- \emph{January '10 -- present}

Expected graduation date: May '15

Fellow of the \textbf{Institute for Quantum Computing}

Recipient of a \textbf{David R. Cheriton Graduate Scholarship},\\awarded
annually to forty to seventy-five full-time University of Waterloo
Computer Science graduate students on the basis of scholastic excellence
and evidence of research potential.

\subsection{M.Math in Computer
Science}\label{m.math-in-computer-science}

\emph{University of Pisa} --- \emph{February '09}

Final score: 110/110 \emph{cum laude}

\subsection{B.Math in Computer
Science}\label{b.math-in-computer-science}

\emph{University of Pisa} --- \emph{July '06}

Final score: 110/110 \emph{cum laude}

\section{Selected Articles}\label{selected-articles}

\begin{itemize}
\itemsep1pt\parskip0pt\parsep0pt
\item
  \textbf{Limitations on separable measurements by convex optimization,
  2014}
\item
  \textbf{Small sets of locally indistinguishable orthogonal maximally
  entangled states, 2014}
\item
  \textbf{PPT-indistinguishable states via semidefinite programming,
  2012}
\end{itemize}

Used \emph{convex programming} to answer several fundamental open
questions in the topic of \emph{quantum state discrimination}.

(Complete list of publications available at
\url{https://cosenal.github.io/papers/})

\section{Work Experience}\label{work-experience}

\subsection{Google Summer of Code student
developer}\label{google-summer-of-code-student-developer}

\emph{KDE} --- \emph{Summer '12}

Created the \href{https://github.com/owncloud/news}{ownCloud News} app,
a feed reader for \href{http://owncloud.org/}{ownCloud}. The app has
been among the top 5 highest rated apps on the
\href{http://apps.owncloud.com/index.php?xsortmode=high}{ownCloud App
store} for two years and it has served as a testbed for ownCloud core
technologies, such as the app framework.

\subsection{Season of KDE student
developer}\label{season-of-kde-student-developer}

\emph{Summer '11}

Built a component for the porting of the KDE feed reader Akregator to
the storage service Akonadi introduced in KDE 4.

\subsection{Teaching assistant}\label{teaching-assistant}

\emph{University of Waterloo} --- \emph{2010 -- 2013}

Courses: Theory of Quantum Information (graduate), Data Structures and
Data Management, Algorithms, Introduction to Computer Science.

\section{Other Experience}\label{other-experience}

\subsection{\href{https://gnome.org/opw/}{Outreach Program for Women}
org coordinator and
mentor}\label{outreach-program-for-womenopw-org-coordinator-and-mentor}

\emph{GNOME Foundation} --- \emph{Summer '14}

Coordinated the first participation of ownCloud at OPW and mentored a
successful ownCloud project.

\subsection{UNIX Consultant}\label{unix-consultant}

\emph{Math Faculty Computing Facility, University of Waterloo} ---
\emph{Winter '12} and \emph{Fall '12}

Assisted students, faculty and staff with computer related problems.

\subsection{\href{https://www.google-melange.com/gci/homepage/google/gci2012}{Google
Code-in} org
administrator}\label{google-code-incodein-org-administrator}

\emph{KDE} --- \emph{Winter '12}

\section{Exchange programs}\label{exchange-programs}

\subsection{Research intern}\label{research-intern}

\emph{LIAFA, Université Paris Diderot} --- \emph{February '13 -- April
'13}

\subsection{\href{http://en.wikipedia.org/wiki/Erasmus_Programme}{Erasmus}
scholar}\label{erasmus-scholar}

\emph{Aarhus University} --- \emph{September '07 -- March '08}

\section{Professional Service}\label{professional-service}

Reviewer for the journals \emph{APS Physical Review A}, \emph{IEEE
Transactions on Information Theory}, and for the \emph{XVII Conference
on Quantum Information Processing (2014)}.

\section{Technical Skills}\label{technical-skills}

\begin{itemize}
\itemsep1pt\parskip0pt\parsep0pt
\item
  Research tools: LaTeX, MATLAB/Octave and framework CVX for convex
  programming;
\item
  Web programming languages: PHP, HTML5, Javascript (with AngularJS
  framework), CSS;
\item
  Other technologies: RSS and Atom standards, Git.
\end{itemize}

\end{document}
